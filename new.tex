\documentclass[14pt]{extarticle}
\usepackage{xcolor}
\usepackage{amsmath}
\usepackage{bbm}
\tolerance 1
\emergencystretch \maxdimen
\hyphenpenalty 10000
\hbadness 10000
\setlength{\parindent}{0pt}

\title{Digital Signature Algorithm}
\date{2024-03-09}
\author{Mattia Biral}

\begin{document}
    \pagenumbering{gobble}
    \maketitle
    \newpage
    \pagenumbering{arabic}
    \sffamily

    \section{Introduzione}

    Il \textbf{Digital Signature Algorithm} è un sistema crittografico a chiave pubblica e uno \emph{standard federale per l'elaborazione delle informazioni}
    La chiave privata è utilizzata per generare la firma, mentre la chiave pubblica per verificarla.

    E' basato sul problema matematico del \underline{logaritmo discreto}.

    \subsection{Firma digitale}

    La firma digitale fornisce:
    \begin{itemize}
        \item Autenticazione: so chi ha inviato il messaggio
        \item Integrità: so che il documento non è stato modificato dopo la firma
        \item Non-ripudio: l'autore non può dire di non essere stato lui a firmare (side-effect dell'autenticazione)
    \end{itemize}

    \subsection{Operazioni}

    DSA si svolge in quattro operazioni:
    \begin{itemize}
        \item Generazione delle chiavi
        \item Distribuzione delle chiavi
        \item Firma
        \item Verifica della firma
    \end{itemize}

    \section{Algoritmo}

    \subsection{Generazione delle chiavi}

    \subsubsection{Parametri}

    I parametri dell'algoritmo sono $(p, q, g)$

    \begin{itemize}
        \item $H$ funzione crittografica di hash di lunghezza $|H|$ bit (se $|H|$ è maggiore della lunghezza del modulo $N$ solo gli $N$ bit più significativi dell'output saranno utilizzati)
        \item $L$ lunghezza della chiave
        \item $N$ lunghezza del modulo tale che $N < L \wedge N \leq  |H| $
        \item $q$ primo di $N$ bit
        \item $p$ primo di $L$ bit tale che $q \mid p-1$
        \item $h$ casuale in $\{2, ..., p-2\} = \mathbb{F}_p^* - \{1, p - 1\}$
        \item $g := h^{{p-1}/q} \mod p$ (se $g=1$ è necessario generare un nuovo $h$)
    \end{itemize}
\end{document}