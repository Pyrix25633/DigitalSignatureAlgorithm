\documentclass[14pt]{extarticle}
\usepackage{xcolor}
\usepackage{amsmath}
\tolerance 1
\emergencystretch \maxdimen
\hyphenpenalty 10000
\hbadness 10000
\setlength{\parindent}{0pt}

\title{Digital Signature Algorithm}
\date{2024-03-09}
\author{Mattia Biral}

\begin{document}
    \pagenumbering{gobble}
    \maketitle
    \newpage
    \pagenumbering{arabic}
    \sffamily

    \section{Introduzione}

    Il \textbf{Digital Signature Algorithm} è un sistema crittografico a chiave pubblica e uno \emph{standard federale per l'elaborazione delle informazioni} per le \textbf{firme digitali}.
    
    La chiave privata è utilizzata per generare la firma, mentre la chiave pubblica per verificarla.

    E' basato sul problema matematico del \underline{logaritmo discreto}.

    \subsection{Firma digitale}

    La firma digitale fornisce:
    \begin{itemize}
        \item Autenticazione: so chi ha inviato il messaggio
        \item Integrità: so che il documento non è stato modificato dopo la firma
        \item Non-ripudio: l'autore non può dire di non essere stato lui a firmare (side-effect dell'autenticazione)
    \end{itemize}

    \subsection{Operazioni}

    DSA si svolge in quattro operazioni:
    \begin{itemize}
        \item Generazione delle chiavi
        \item Distribuzione delle chiavi
        \item Firma
        \item Verifica della firma
    \end{itemize}

    \section{Algoritmo}

    \subsection{Generazione delle chiavi}

    \subsubsection{Parametri}

    I parametri dell'algoritmo sono $(p, q, g)$

    \begin{itemize}
        \item $H$ funzione crittografica di hash di lunghezza $|H|$ bit (se $|H|$ è maggiore della lunghezza del modulo $N$ solo gli $N$ bit più significativi dell'output saranno utilizzati)
        \item $L$ lunghezza della chiave
        \item $N$ lunghezza del modulo tale che $N < L \wedge N \leq  |H| $
        \item $q$ primo di $N$ bit
        \item $p$ primo di $L$ bit tale che $q \mid p-1$
        \item $h$ casuale in $\{2, ..., p-2\}$
        \item $g := h^{{p-1}/q} \mod p$ (se $g=1$ è necessario generare un nuovo $h$)
    \end{itemize}

    \subsubsection{Chiavi per-user}

    \begin{itemize}
        \item $x$ casuale in $\{1, ..., q-1\}$, \textbf{chiave privata}
        \item $y := g^x \mod p$, \textbf{chiave pubblica}
    \end{itemize}

    \subsection{Distribuzione delle chiavi}

    Il firmatario pubblica la chiave \emph{pubblica} $y$ e mantiene segreta $x$

    \subsection{Firma}

    Un messaggio $m$ è firmato come segue:
    \begin{itemize}
        \item $k$ casuale in $\{1, ..., q-1\}$
        \item $r := (g^k \mod p) \mod q = g^k \mod pq$ (se $r = 0$ è necessario generare un nuovo $k$)
        \item $s := (k^{-1}(H(m) + xr)) \mod q$ (se $s = 0$ è necessario scegliere un'altro $k$)
    \end{itemize}
    La firma è $(r, s)$

    \bigskip
    Nota: dato che ogni volta che si firma il messaggio viene scelto un $k$ casuale, è molto probabile che più firme di uno stesso documento siano \textbf{diverse ma ugualmente valide}.

    \bigskip
    Nota: la computazione più costosa riguarda $r$, ma può essere fatta prima di conoscere $m$ poiché non dipende da esso.

    \bigskip
    Nota: la seconda computazione più costosa riguarda $k^{-1}$, ma anch'essa può essere effettuata prima che $m$ sia noto.

    \subsection{Verifica della firma}

    Si verifica che una firma è autentica come segue:
    \begin{itemize}
        \item $0 < r < q \wedge 0 < s < q$ (in quanto risultati di $... \mod q$)
        \item $w := s^{-1} \mod q$
        \item $u_1 := H(m) \cdot w \mod q$
        \item $u_2 := r \cdot w \mod q$
        \item $v := (g^{u_1}y^{u_2} \mod p) \mod q$
        \item $v = r$
    \end{itemize}

    \section{Correttezza dell'algoritmo}

    Lo schema di firma è corretto nel senso che il verificatore accetterà sempre firme autentiche.

    \bigskip
    Dato che
    
    \begin{equation*}
        g = h^{(p-1)/q} \mod p
        \implies
        g^q \equiv h^{p-1} \equiv 1 \mod p
    \end{equation*}
    per il teorema di Eulero.
    Poiché $g > 0$ e $g$ è primo, $g$ ha ordine $q$.

    \bigskip
    Il firmatario calcola
    \begin{equation*}
        s = k^{-1}(H(m) + xr) \mod p
    \end{equation*}
    Quindi
    \begin{align*}
        k &\equiv H(m)s^{-1} + xrs^{-1}\\
          &\equiv H(m)w + xrw \mod q
    \end{align*}
    Dal momento che $g$ ha ordine $q$ abbiamo
    \begin{align*}
        g^k &\equiv g^{H(m)w}g^{xrw}\\
            &\equiv g^{H(m)w}y^{rw}\\
            &\equiv g^{u_1}y^{u_2} \mod p
    \end{align*}
    Infine
    \begin{align*}
        r &= (g^k \mod p) \mod q\\
          &= (g^{u_1}y^{u_2} \mod p) \mod q\\
          &= v
    \end{align*}

\end{document}